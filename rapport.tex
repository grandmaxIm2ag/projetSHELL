\documentclass{report}
\usepackage[latin1]{inputenc}
\usepackage[T1]{fontenc}
\usepackage[francais]{babel}
\usepackage{lmodern}
\usepackage{amsmath}
\usepackage{amssymb}
\usepackage{mathrsfs}
\usepackage{mathrsfs}
\usepackage{textcomp}
\usepackage{listings}
\usepackage{siunitx}
\lstset{
  basicstyle=\footnotesize\tt,        % the size of the fonts that are used for the code
  breakatwhitespace=false,         % sets if automatic breaks should only happen at whitespace
  breaklines=true,                 % sets automatic line breaking
  captionpos=b,                    % sets the caption-position to bottom
  extendedchars=true,              % lets you use non-ASCII characters; for 8-bits encodings only, does not work with UTF-8
  frame=single,                    % adds a frame around the code
  language=C,                 % the language of the code
  keywordstyle=\bf,
  showspaces=false,                % show spaces everywhere adding particular underscores; it overrides 'showstringspaces'
  showstringspaces=false,          % underline spaces within strings only
  showtabs=false,                  % show tabs within strings adding particular underscores
  tabsize=2                       % sets default tabsize to 2 spaces
}





\author{Grand Maxence, Muller Lucie}
\title{Syst\`emes et R\'eseaux  : R\'ealisations d'un mini shell}
\date{17/02/2017}


\begin{document}
	
	\maketitle
	\tableofcontents

	\chapter*{Introduction}
	Durant le projet, nous avons cr\'e\'e un interpr\^eteur de commande shell mais d'une fa\c{c}on simplifi\'ee. Il a fallut en comprendre la structure pour la refaire, et permettre a notre programme d'acc\`eder \`a des commandes simples puis \`a des commandes plus complexes g\`erant les redirections d'entr\'ees-sorties ainsi que les s\'equences de commandes pip\'ees.\\
	Dans un second temps, nous avons am\'elior\'e notre shell pour lui permettant la gestion des diff\'erents plans, des signaux de terminaison et de suspension, des zombis et des jobs.
	\chapter{Partie 1}
		\section{Commandes simples}
		Dans cette partie nous devions impl\'ement\'e une fonction permettant d'ex\'ecuter une commande simple en shell. Pour tester no\^otre fonction nous avons tester plusieurs commande n'utilisant ni les pipes ni les redirections dans un shell et dans n\^otre programme shell. \\
		Code :
		\begin{lstlisting}
/*
Cette fonction permet d'executer une commande simple sans redirection et sans tube
*/
void commande_simple(struct cmdline *l){
	//Processus pere cree un processus fils qui devra executer la commande
	int pid = Fork();
	int status;
	if(pid == 0){
		/*Si on est le fils alors
			on execute la commande passe en paramettre
			on s'arrette
		*/
		execvp(l->seq[0][0], l->seq[0]);
		exit(0);
	}else{
		/*
		Le pere attend la fin de son fils
		*/
		while(waitpid(pid, &status, 0) != pid);
	}
}
		\end{lstlisting} 
		Test \\ SHELL
		\begin{lstlisting}[frame=single,basicstyle=\footnotesize,language=bash]
maxence@Sybil:~ ls
csapp.c  Makefile     rapport.log  rapport.tex  readcmd.c  README.md  shell.c    test   test3  tst    tube_simple.c
csapp.h  rapport.aux  rapport.pdf  rapport.toc  readcmd.h  shell      sujet.pdf  test2  text   tst.c  waitpid1.c
maxence@Sybil:~ ls -l
total 2688
-rwx------ 1 maxence maxence  20259 déc.  25  2014 csapp.c
-rwx------ 1 maxence maxence   6105 mars  16  2014 csapp.h
-rwx------ 1 maxence maxence    136 oct.   6  2009 Makefile
-rwx------ 1 maxence maxence   1499 févr.  8 15:58 rapport.aux
-rwx------ 1 maxence maxence  30949 févr.  8 15:58 rapport.log
-rwx------ 1 maxence maxence 128494 févr.  8 15:58 rapport.pdf
-rwx------ 1 maxence maxence   2558 févr.  8 16:02 rapport.tex
-rwx------ 1 maxence maxence    933 févr.  8 15:58 rapport.toc
-rwx------ 1 maxence maxence   4699 févr.  8 14:07 readcmd.c
-rwx------ 1 maxence maxence   1029 févr.  8 14:02 readcmd.h
-rwx------ 1 maxence maxence     14 févr.  7 18:02 README.md
-rwx------ 1 maxence maxence  41024 févr.  8 15:45 shell
-rwx------ 1 maxence maxence   4338 févr.  8 16:03 shell.c
-rwx------ 1 maxence maxence  94332 févr.  7 12:01 sujet.pdf
-rwx------ 1 maxence maxence    150 févr.  8 13:57 test
-rwx------ 1 maxence maxence      3 févr.  8 13:58 test2
-rwx------ 1 maxence maxence      3 févr.  7 15:51 test3
-rwx------ 1 maxence maxence      0 févr.  7 15:36 text
-rwx------ 1 maxence maxence  18008 févr.  8 12:08 tst
-rwx------ 1 maxence maxence    723 févr.  7 13:55 tst.c
-rwx------ 1 maxence maxence    815 janv. 31 15:18 tube_simple.c
-rwx------ 1 maxence maxence    657 janv. 20  2009 waitpid1.c
maxence@Sybil:~ cat test
19
pp.c
csapp.h
Makefile
readcmd.c
readcmd.h
README.md
shell
shell.c
shell.o
sujet.pdf
test
test2
test3
text
tst
tst.c
tst.o
tube_simple.c
waitpid1.c
		\end{lstlisting}
		shell.c
		\begin{lstlisting}[frame=single,basicstyle=\footnotesize,language=bash]
shell> ls
csapp.c  Makefile     rapport.log  rapport.tex	readcmd.c  README.md  shell.c	 test	test3  tst    tube_simple.c
csapp.h  rapport.aux  rapport.pdf  rapport.toc	readcmd.h  shell      sujet.pdf  test2	text   tst.c  waitpid1.c
shell> ls -l
total 2688
-rwx------ 1 maxence maxence  20259 déc.  25  2014 csapp.c
-rwx------ 1 maxence maxence   6105 mars  16  2014 csapp.h
-rwx------ 1 maxence maxence    136 oct.   6  2009 Makefile
-rwx------ 1 maxence maxence   1499 févr.  8 15:58 rapport.aux
-rwx------ 1 maxence maxence  30949 févr.  8 15:58 rapport.log
-rwx------ 1 maxence maxence 128494 févr.  8 15:58 rapport.pdf
-rwx------ 1 maxence maxence   4558 févr.  8 16:05 rapport.tex
-rwx------ 1 maxence maxence    933 févr.  8 15:58 rapport.toc
-rwx------ 1 maxence maxence   4699 févr.  8 14:07 readcmd.c
-rwx------ 1 maxence maxence   1029 févr.  8 14:02 readcmd.h
-rwx------ 1 maxence maxence     14 févr.  7 18:02 README.md
-rwx------ 1 maxence maxence  41024 févr.  8 16:06 shell
-rwx------ 1 maxence maxence   4338 févr.  8 16:03 shell.c
-rwx------ 1 maxence maxence  94332 févr.  7 12:01 sujet.pdf
-rwx------ 1 maxence maxence    150 févr.  8 13:57 test
-rwx------ 1 maxence maxence      3 févr.  8 13:58 test2
-rwx------ 1 maxence maxence      3 févr.  7 15:51 test3
-rwx------ 1 maxence maxence      0 févr.  7 15:36 text
-rwx------ 1 maxence maxence  18008 févr.  8 12:08 tst
-rwx------ 1 maxence maxence    723 févr.  7 13:55 tst.c
-rwx------ 1 maxence maxence    815 janv. 31 15:18 tube_simple.c
-rwx------ 1 maxence maxence    657 janv. 20  2009 waitpid1.c
shell> cat test
19
pp.c
csapp.h
Makefile
readcmd.c
readcmd.h
README.md
shell
shell.c
shell.o
sujet.pdf
test
test2
test3
text
tst
tst.c
tst.o
tube_simple.c
waitpid1.c
shell>  echo "test"
"test"
shell>
		\end{lstlisting}
		\section{Commande avec redirections d'entr\'ee ou de sortie}
			Dans cette section nous devions implenter une fonction permettant de ger\'les redirections d'entr\'ee ou de sortie. Nos test ont suivi les m\^eme principes que l'\'etape pr\'ec\'edente \`a l'exception que seul les commandes avec pipes ne seront pas test\'ees ici. \\ Code
			\begin{lstlisting}
/*
Cette fonction permet d'executer une commande avec redirection des flux sans pipe
*/
void commande_redirection(struct cmdline *l){
	/*
	fIn : le descripteur de fichier d'entrer, initialiser a 1, l'entree standard
	fOut : le descripteur de fichier de sortie, initialiser a 1, le sortie standard
	*/
	int fOut = 1; int fIn = 0;

	if(l->in != NULL)
		/*
		Si, ete present, '> <fic>' alors fIn prend la valeur du descripteur de <fic>
		*/
		fIn = open(l->in, O_RDONLY,0);
	if(l->out != NULL)
		fOut = open(l->out, O_WRONLY | O_CREAT, 0700);
		/*
		Si, ete present, '< <fic>' alors fOut prend la valeur du descripteur de <fic>
		*/
	int pid = Fork(); int status;
	if(pid == 0){
		if(fOut != 1){
			/*
			Si fOut diffrenrent de 1 alors
				On ferme la sortie standard
				La sortie de l'execution devient le fichier descrit par fOut
			*/
			close(1);
			dup2(fOut, 1);
		}

		/*
			Si fIn diffrenrent de i alors
				On ferme l'entree standard
				L'entree de l'execution devient le fichier descrit par fIn
				On execute
				On s'arrete
			Sinon
				On execute
				On s'arrete
			*/
		if(fIn == 0){
			execvp(l->seq[0][0], l->seq[0]);
			exit(0);
		}else{
			close(0);
			dup2(fIn, 0);
			execvp(l->seq[0][0], l->seq[0]);
			exit(0);
		}
		
	}
	else{
		/*
		Le pere attend la fin d'execution du fils
		*/
		if(l->bg){
				ajout_job(l, pid, 0);
			}else{
				child = pid;
				Signal(SIGINT, stop);	
				Signal(SIGTSTP, suspend);
				ajout_job(l, pid, 1);
				//printf("creer\n");
				while(waitpid(pid, &status, WUNTRACED) != pid);
				if(getStatus(pid) != 2 && getStatus(pid)!=-1){
					supprime_job(pid);
				}
				//printf("supprimer\n");
			}
	}
}
			\end{lstlisting}
			Test \\ SHELL
			\begin{lstlisting}[frame=single,basicstyle=\footnotesize,language=bash]
maxence@Sybil:~ ls -l > test
maxence@Sybil:~ cat test
total 2688
-rwx------ 1 maxence maxence  20259 déc.  25  2014 csapp.c
-rwx------ 1 maxence maxence   6105 mars  16  2014 csapp.h
-rwx------ 1 maxence maxence    136 oct.   6  2009 Makefile
-rwx------ 1 maxence maxence   1499 févr.  8 16:10 rapport.aux
-rwx------ 1 maxence maxence  31866 févr.  8 16:10 rapport.log
-rwx------ 1 maxence maxence 218712 févr.  8 16:10 rapport.pdf
-rwx------ 1 maxence maxence   7408 févr.  8 16:16 rapport.tex
-rwx------ 1 maxence maxence    933 févr.  8 16:10 rapport.toc
-rwx------ 1 maxence maxence   4699 févr.  8 14:07 readcmd.c
-rwx------ 1 maxence maxence   1029 févr.  8 14:02 readcmd.h
-rwx------ 1 maxence maxence     14 févr.  7 18:02 README.md
-rwx------ 1 maxence maxence  41024 févr.  8 16:06 shell
-rwx------ 1 maxence maxence   4338 févr.  8 16:03 shell.c
-rwx------ 1 maxence maxence  94332 févr.  7 12:01 sujet.pdf
-rwx------ 1 maxence maxence      0 févr.  8 16:16 test
-rwx------ 1 maxence maxence      3 févr.  8 13:58 test2
-rwx------ 1 maxence maxence      3 févr.  7 15:51 test3
-rwx------ 1 maxence maxence      0 févr.  7 15:36 text
-rwx------ 1 maxence maxence  18008 févr.  8 12:08 tst
-rwx------ 1 maxence maxence    723 févr.  7 13:55 tst.c
-rwx------ 1 maxence maxence    815 janv. 31 15:18 tube_simple.c
-rwx------ 1 maxence maxence    657 janv. 20  2009 waitpid1.c
maxence@Sybil:~ wc -l < test
23
maxence@Sybil:~ wc -c < test
1347
maxence@Sybil:~ wc - < test > test2
maxence@Sybil:~ cat test2
  23  200 1347 -
		\end{lstlisting}
		shell.c
		\begin{lstlisting}[frame=single,basicstyle=\footnotesize,language=bash]
shell> ls -l > test
shell> cat test
total 2816
-rwx------ 1 maxence maxence  20259 déc.  25  2014 csapp.c
-rwx------ 1 maxence maxence   6105 mars  16  2014 csapp.h
-rwx------ 1 maxence maxence    136 oct.   6  2009 Makefile
-rwx------ 1 maxence maxence   1499 févr.  8 16:10 rapport.aux
-rwx------ 1 maxence maxence  31866 févr.  8 16:10 rapport.log
-rwx------ 1 maxence maxence 218712 févr.  8 16:10 rapport.pdf
-rwx------ 1 maxence maxence   9314 févr.  8 16:22 rapport.tex
-rwx------ 1 maxence maxence    933 févr.  8 16:10 rapport.toc
-rwx------ 1 maxence maxence   4699 févr.  8 14:07 readcmd.c
-rwx------ 1 maxence maxence   1029 févr.  8 14:02 readcmd.h
-rwx------ 1 maxence maxence     14 févr.  7 18:02 README.md
-rwx------ 1 maxence maxence  41024 févr.  8 16:23 shell
-rwx------ 1 maxence maxence   4343 févr.  8 16:23 shell.c
-rwx------ 1 maxence maxence  94332 févr.  7 12:01 sujet.pdf
-rwx------ 1 maxence maxence      1 févr.  8 16:17 test
-rwx------ 1 maxence maxence      1 févr.  8 16:17 test2
-rwx------ 1 maxence maxence      3 févr.  7 15:51 test3
-rwx------ 1 maxence maxence      0 févr.  7 15:36 text
-rwx------ 1 maxence maxence  18008 févr.  8 12:08 tst
-rwx------ 1 maxence maxence    723 févr.  7 13:55 tst.c
-rwx------ 1 maxence maxence    815 janv. 31 15:18 tube_simple.c
-rwx------ 1 maxence maxence    657 janv. 20  2009 waitpid1.c
shell> wc -l < test
23
shell> wc -c < test
1347
shell> wc - < test > test2
shell> cat test2
  23  200 1347 -
		\end{lstlisting}
		\section{S\'equence de commandes compos\'ee de deux commandes reli\'es par un tube}
		Dans cette section nous devions implanter une fonction permettant de g\'erer les commandes reli\'es par un ou plusieurs tubes. Nos test ont suivi les m\^eme principes que l'\'etape pr\'ec\'edente \`a l'exception que seul les commandes avec redirections ne seront pas test\'ees ici. \\ Code
			\begin{lstlisting}
/*
Cette fonction permet d'executer une suite d'instruction pipee sans redirection de flus d'entrees/sorties
*/
void commande_pipe(struct cmdline *l){
	int tailleSeq; int tmp;
	//On calcul le nb de commande pipee
	for(tailleSeq=0; l->seq[tailleSeq+1]!=0; tailleSeq++);
	int pid = Fork(); int status; int i=0; int desc[2];
	if(pid == 0){
		/*
		Le premier fils creer pipe qu'il partageras avec son futur fil
		Il ferme son entrer standard et recupere comme entree la sortie du pipe
		Attend la fin de son fils
		s'execute
		s'arrete
		*/
		pipe(desc);
		pid=Fork();
		if(pid != 0){
			dup2(desc[0], 0);
			close(desc[1]);
			while( (tmp = waitpid(pid, &status, WNOHANG|WUNTRACED)) != pid);
			execvp(l->seq[1][0], l->seq[tailleSeq]);
			close(desc[0]);
			exit(0);
		}else{
			/*
			Tant qu'il reste des commandes suivantes
				Si nous sommes la derniere commande
					On ferme sa sortie standard et recupere comme sortie l'entree du pipe partage avec le pere
					on cree un pipe
					On cree un processus fils
					Si on est le pere
						On ferme  son entrer standard et recupere comme entree la sortie du pipe
						on attend la fin de son fils
						On s'execute
						On s'arrete
				Sinon
					On ferme sa sortie standard et recupere comme sortie l'entree du pipe partage avec le pere
					on cree un pipe
					On s'execute
					On s'arrete
			*/
			for(i = 1; i<=tailleSeq; i++){
				if(i+1 <= tailleSeq){
					dup2(desc[1], 1);
					close(desc[0]);
					pipe(desc);
					pid = Fork();
					if(pid != 0){
						dup2(desc[0], 0);
						close(desc[1]);
						while(waitpid(pid, &status, 0) != pid);
						execvp(l->seq[tailleSeq - i][0], l->seq[tailleSeq-1]);
						exit(0);
					}
				} else{
					dup2(desc[1], 1);
					close(desc[0]);
					execvp(l->seq[0][0], l->seq[0]);
					exit(0);
				}
			}		
		}
	}else{
		// le processus pere attend la fin de tous ses descendants.
		while(waitpid(pid, &status, 0) != pid);
	}
}
			\end{lstlisting}
			Test \\ SHELL
			\begin{lstlisting}[frame=single,basicstyle=\footnotesize,language=bash]
shell> ls -l | wc -l
23
shell> ls -l | wc -l | wc -c
3
			\end{lstlisting}
			shell.c
			\begin{lstlisting}[frame=single,basicstyle=\footnotesize,language=bash]
maxence@Sybil:~ ls -l | wc -l
23
maxence@Sybil:~ ls -l | wc -l | wc -c
3
			\end{lstlisting}
		\section{S\'equence de commandes compos\'ee de plusieurs commandes et plusieurs tubes}
		Dans cette section nous devions implanter une fonction permettant de g\'erer les commandes reli\'es par un ou plusieurs tubes et avec redirections. Nos test ont suivi les m\^eme principes que l'\'etape pr\'ec\'edentetous les types de commandes sont test\'ees ici. \\ Code
			\begin{lstlisting}
/*
Cette fontion permet d'executer une sequence de commandes pipees avec redirection de flus d'entrees/sorties 
*/
void commande1_final(struct cmdline *l){
	if(l->seq[1]!=0){
		/*
		SI nous avons une sequences de commande pipe alors
			On gere les les flux d'entree/sortie comme pour la fonction commande_redirection
			La suite est gerer de la meme facon que dans fonction commande_pipe
		*/
		int fOut = 1; int fIn = 0;
			if(l->in != NULL)
			fIn = open(l->in, O_RDONLY,0);
		if(l->out != NULL)
			fOut = open(l->out, O_WRONLY | O_CREAT, 0700) ;
		commande_pipe(l,fOut, fIn );
	}else{
		/*Si nous n'avons pas de commande pipees alors
			nous appelons la fonction commande_redirection
		*/ 
		commande_redirection(l);
	}
}

			\end{lstlisting}
			Test \\ SHELL
			\begin{lstlisting}[frame=single,basicstyle=\footnotesize,language=bash]
maxence@Sybil:~ head -5 < test | tail -2 | sort > tmp2
maxence@Sybil:~ cat tmp2
-rwx------ 1 maxence maxence    136 oct.   6  2009 Makefile
-rwx------ 1 maxence maxence   1499 févr.  8 16:10 rapport.aux
			\end{lstlisting}
			shell.c
			\begin{lstlisting}[frame=single,basicstyle=\footnotesize,language=bash]
shell>  head -5 < test | tail -2 | sort > tmp
shell> cat tmp
-rwx------ 1 maxence maxence    136 oct.   6  2009 Makefile
-rwx------ 1 maxence maxence   1499 févr.  8 16:10 rapport.aux
			\end{lstlisting}
	\chapter{Partie Bonus}
		\section{Ex\'ecution de commandes en arri\`ere plan}
			Dans cette section, nous avons impl\'ement\'e une fonction permettant de faire fonctionner des processus en arri\`ere plan : ils continuent de fonctionner sans bloquer le shell, ce qui permet a l'utilisateur de continuer de travailler et taper d'autres commandes pendant que le processus continue.\\Code :
			\begin{lstlisting}
/*
Cette fonction permet de gerer les commandes avec &
*/
void commande_bg(struct cmdline *l){
	/*
	Si la commande est suivi d'un & alors
		Alors la commande est en second plan et le pere n'attend pas la fin de l'execution de son fils
	SInon
		Le pere attend la fin de l'execution de son fils
	*/ 
	int pid, status;

	pid = Fork();

	if(pid == 0){
		execvp(l->seq[0][0], l->seq[0]);
		exit(0);
	}else{
		if(l->bg);
		else{
			waitpid(pid, &status, 0);
		}
	}
}
	\end{lstlisting}			
	Test \\ SHELL :
	\begin{lstlisting}[frame=single,basicstyle=\footnotesize,language=bash]
maxence@Sybil:~ kate &
[1] 7381
maxence@Sybil:~ kate

			\end{lstlisting}
			shell.c
			\begin{lstlisting}[frame=single,basicstyle=\footnotesize,language=bash]
shell> kate &
7347
shell> kate
7362

			\end{lstlisting}
		\section{Changer l'\'etat du processus au premier plan}
	Dans cette section, on a modifi\'e notre interpr\^eteur afin qu'il reconnaisse certains signaux agissant sur les processus en premier plan qui sont les signaux de terminaison et de suspension respectivement envoy\'es par la pression des touches Ctrl-C et Ctrl-Z.\\Code :
			\begin{lstlisting}
void stop(int sig){
	printf("Fini\n");
	kill(child, SIGKILL);
}
void suspend(int sig){
	printf("Suspendu\n");
	kill(child, SIGSTOP);
}
/*
Cette fonction permet de gerer les sinaux SIGINT et SIGSTSTP
*/
void commande_signaux(struct cmdline *l){

	int pid = Fork();
	int status;
	Signal(SIGINT, stop);	
	Signal(SIGTSTP, suspend);

	if(pid == 0){
		execvp(l->seq[0][0], l->seq[0]);
		exit(0);
	}else{
		child = pid;
		while(waitpid(pid, &status, WUNTRACED) != pid);
		//Le pere attend un signal tant que son fils n'a pas fini son execution
	}
}			
			\end{lstlisting}
			Test \\ SHELL :
			\begin{lstlisting}[frame=single,basicstyle=\footnotesize,language=bash]
maxence@Sybil:~ gedit 
^C
[2]+  Fini                    okular
maxence@Sybil:~ gedit 
^Z
[1]+  Arrêté                gedit

			\end{lstlisting}
			shell.c
			\begin{lstlisting}[frame=single,basicstyle=\footnotesize,language=bash]
shell> kate
7498
^Z
Suspendu 7498
shell> kate
7514
^Z
Suspendu 7514
			\end{lstlisting}
		\section{Gestion des zombis}
		Dans cette section nous devons g\`erer les cas de prosessus fils qui n'ont pas termin\'es lors de la terminaison du shell, appel\'es zombis. Pour cela nous avons ajout\'e un signal qui le fait attendre jusqu'\`a la terminaison de chacun de ces fils, avant de se terminer lui-m\^eme.\\Code :\\
			\begin{lstlisting}
void zombi(int sig)
{
	/*
	Attend la fin des processus zombis
	*/
  	waitpid(-1, NULL, WNOHANG|WUNTRACED);
}

void commande_zombi(struct cmdline *l){
	Signal( SIGCHLD, zombi);
	int pid = Fork();
	int status;
	if(pid == 0){
		execvp(l->seq[0][0], l->seq[0]);
		exit(0);
	}
}
			\end{lstlisting}
		\section{Commande integr\'ee jobs}
			Dans cette section, nous avons impl\'ementer la fonction Jobs du shell permettant d'afficher sur la sortie standard les differents job d\'emarr\'es durant la session courante de ce shell. Quand jobs indique la terminaison d'un job, il l'efface de la liste et donc ce job n'est plus dans la liste \`a l'appel suivant de jobs.\\Code :
			\begin{lstlisting}
/*
Structure representant les jobs
*/
struct job{
	char* cmd; //Commande du processus
	int pid; //pid du processus
	int status; //status du job (bg, fg suspend)
};

/*
fonction permettant de gerer l'ajout de processus dans les jobs
*/
void ajout_job(struct cmdline *l, int pid, int status){
	/*
	Alloue ou realloue la memoire pour notre structure job
	*/
	if(nbEnCours == 0){
		enCours = malloc(sizeof(struct job*)*(nbEnCours+1));
		enCours[nbEnCours] = malloc(sizeof(struct job)*1);
	}else{
		enCours = realloc(enCours, (sizeof(struct job*)*(nbEnCours+1)));
		enCours[nbEnCours] = malloc(sizeof(struct job)*1);
	}
	/*Copie du nom de la commande du processus*/ 
	enCours[nbEnCours]->cmd = malloc(sizeof(char)*strlen(l->seq[0][0]));
	memcpy(enCours[nbEnCours]->cmd, l->seq[0][0], strlen(l->seq[0][0]));
	/*
	Copie du numéro de pid et du status
	*/
	enCours[nbEnCours]->pid = pid;
	enCours[nbEnCours]->status = status;
	nbEnCours++;
}

/*
Affiche les jobs en cours
*/
void affiche_job(){
		int i;
		for(i=0; i<nbEnCours; i++){
			switch(enCours[i]->status){
				case 0:
					printf("[%d] \t En cours d'execution en arriere plan \t%d \t%s\n", i+1,enCours[i]->pid,  enCours[i]->cmd);
					break;
				case 1:
					printf("[%d] \t En cours d'execution au premier plan\t%d\t %s\n", i+1,enCours[i]->pid, enCours[i]->cmd);
					break;
				default:
					printf("[%d] \t Suspendu \t%d\t %s\n", i+1, enCours[i]->pid, enCours[i]->cmd);
			}
		}
}

/*
Fonction permettant de gerer l'ajout et l'affichage de job
*/
void commande_job(struct cmdline *l){
	int pid, status;

	pid = Fork();
	Signal(SIGINT, stop);	
	Signal(SIGTSTP, suspend);
	if(pid == 0){
		execvp(l->seq[0][0], l->seq[0]);
		exit(0);
	}else{
		if(l->bg){
			ajout_job(l, pid, 0);
		}
		else{
			child = pid;
			ajout_job(l, pid, 1);
			while(waitpid(pid, &status, WUNTRACED) != pid);
			supprime_job(pid);
		}
	}
}

			\end{lstlisting}
			Test \\
			SHELL : 
			\begin{lstlisting}[frame=single,basicstyle=\footnotesize,language=bash]
maxence@Sybil:~$ jobs
maxence@Sybil:~$ kate &
[1] 7769
maxence@Sybil:~$ jobs
[1]+  En cours d'exécution   kate &
maxence@Sybil:~$ 
			\end{lstlisting}
			shell.c :
			\begin{lstlisting}[frame=single,basicstyle=\footnotesize,language=bash]
shell> jobs
shell> kate &
shell>
jobs
[1] 	 En cours d'execution en arriere plan 	7871 	kate
shell> 
			\end{lstlisting}
		\section{Agir sur les commandes en arri\`ere plan}
		Dans cette section, il fallait que le programme agisse sur les jobs en arri\'ere plan. Pour cela il y avait trois commandes a impl\'ementer qui sont fg pour ramener un jobs au premier plan, bg pour mettre un jobs en arri\`ere plan et stop pour arr\^eter le jobs.\\Code : 
			\begin{lstlisting}
/*
Fonction permettant de supprimer de Pid pid
*/
void supprime_job(int pid){
	printf("delete %d\n",pid);
	/*
	Si il y a au moins 2 jobs, alors on modifie enCours pour qu'il n'y plus le job de Pid pid
	dans la structure enCours 
	*/
	if(nbEnCours>1){
		struct job** tmp = (struct job**)malloc(sizeof(struct job *)*(nbEnCours-1));
		int i;
		int j = 0;
		for(i=0; i<nbEnCours; i++){
			if(enCours[i]->pid != pid){
				tmp[j] = malloc(sizeof(struct job)*1);
				memcpy(tmp[j], enCours[i],sizeof(struct job));
				tmp[j]->cmd = malloc(sizeof(char)*strlen(enCours[i]->cmd));
				memcpy(tmp[j]->cmd,enCours[i]->cmd,sizeof(char)*strlen(enCours[i]->cmd));
				j++;
			}
		}
		if(j==nbEnCours-1){
			nbEnCours--;
			free_job(enCours, nbEnCours);
			enCours = malloc(sizeof(struct job *)*nbEnCours );
			memcpy(enCours,tmp,sizeof(struct job *)*nbEnCours );
		}else{
			free_job(tmp, j);
		}
	/*
	Si il n'y a qu'un seul job dans enCours
	Alors ont libere la memoire allouee pour enCours
	*/
	}else if(nbEnCours == 1){
		free_job(enCours, nbEnCours);
		nbEnCours = 0;
	}else;
}

/*
Fonction permettant de mettre a jours le status des jobs
*/
void maj_job(int pid, int action){
	if(nbEnCours == 0)
		printf("Aucun job en cours\n");
	else{
		int k;
		switch(action){
			case 0: //bg
				/*
				Si la commande bg a ete utilisee alors
					le processus doit s'executer au second plan
					on envoie doonc un signal SIGCONT pour reveiller le prcessus
					Le processus n'attend pas la fin du processus pour reprendre la main
				*/
				for(k = 0; k<nbEnCours && enCours[k]->pid != pid; k++);
				if(k<nbEnCours){
					printf("processus %d mis au second plan\n", pid);
					kill(pid, SIGCONT);
					enCours[k]->status = action;
				}
				else
					printf("Il n'y pas de processus dont le pid est : %d\n",pid);
				break;
			case 1: //fg
				/*
				Si la commande fg a ete utilisee alors
					le processus doit s'executer au premier plan
					on envoie doonc un signal SIGCONT pour reveiller le prcessus
					Le processus pere attend pas la fin du processus pour reprendre la main
				*/
				for(k = 0; k<nbEnCours && enCours[k]->pid != pid; k++);
				if(k<nbEnCours){
					child = pid;
					printf("processus %d mis au premier plan\n", pid);
					kill(pid, SIGCONT);
					enCours[k]->status = action;
					affiche_job();
					int tmp = waitpid(pid, NULL, WUNTRACED);
					printf("test\n");
					while(tmp != pid){
						tmp = waitpid(pid, NULL, WUNTRACED);
					}
					if(getStatus(pid) != 2 && getStatus(pid) != -1)
						supprime_job(pid);
				}	
				else
					printf("Il n'y pas de processus dont le pid est : %d\n",pid);
				break;
			case 2: //stop
			/*
				Si la commande stop a ete utilisee alors
					le processus doit etre suspendu
					on envoie doonc un signal SIGSTOP pour suspendre le prcessus
				*/
				for(k = 0; k<nbEnCours && enCours[k]->pid != pid; k++);
				if(k<nbEnCours){
					printf("processus %d Suspendu\n", pid);
					kill(pid, SIGSTOP);
					enCours[k]->status = action;
				}				
				else
					printf("Il n'y pas de processus dont le pid est : %d\n",pid);
				break;
			default:
				printf("Action impossible\n");
		}
	}
}
void maj_job2(int idx, int action){
	if(nbEnCours == 0)
		printf("Aucun job en cours\n");
	else
		if(idx <=0 || idx > nbEnCours)
			printf("Action impossible, l'indice du pid doit etre en 1 et %d\n",nbEnCours);
		else{
			int pid = enCours[idx-1]->pid;
			maj_job(pid, action);
		}
}

/*
Fonction permettant de gerer les fonctions en premier/second plan ou suspendu, et les jobs.
*/
void commande(struct cmdline *l){
	/*
	Attente de signaux
	*/
	Signal(SIGCHLD, zombi);
	Signal(SIGINT, stop);	
	Signal(SIGTSTP, suspend);
	/*
	Permet de gerer les commandes fg, bg, stop.
	C'est le preocessus pere qui s'occupe de mettre a jour les jobs
	*/
	if((strcmp(l->seq[0][0], "bg") == 0) || (strcmp(l->seq[0][0], "fg") == 0) || (strcmp(l->seq[0][0], "stop") == 0)){
		if(l->seq[0][1][0] == '%'){
			char tmp[strlen(l->seq[0][1])];
			memcpy(tmp, l->seq[0][1]+1, 4);
			int idx = atoi(tmp);
			if(strcmp(l->seq[0][0], "bg") == 0){
				maj_job2(idx, 0);
			}else if(strcmp(l->seq[0][0], "fg") == 0){
				maj_job2(idx, 1);
			}else{
				maj_job2(idx, 2);
			}
		}else{
			int pid = atoi(l->seq[0][1]);
			if(strcmp(l->seq[0][0], "bg") == 0){
				maj_job(pid, 0);
			}else if(strcmp(l->seq[0][0], "fg") == 0){
				maj_job(pid, 1);
			}else{
				maj_job(pid, 2);
			}
		}
	}else if(l->seq[1]!=0){
		/*
		Si nous avons une sequences de commande pipe alors
		On utilise la fonction commande_pipe modifié pour erer l'ajout/suppression de job
		*/

		int fOut = 1; int fIn = 0;
			if(l->in != NULL)
			fIn = open(l->in, O_RDONLY,0);
		if(l->out != NULL)
			fOut = open(l->out, O_WRONLY | O_CREAT, 0700) ;

		commande_pipe(l,fOut, fIn );

	}
	else{
		/*Si nous n'avons pas de commande pipees alors
			nous appelons la fonction commande_redirection modifié pour pouvoir gerer l'jout/suppression de jobs
		*/ 
		commande_redirection(l);
	}
}
			\end{lstlisting}
			Test \\ SHELL :
			\begin{lstlisting}[frame=single,basicstyle=\footnotesize,language=bash]
maxence@Sybil:~$ jobs
maxence@Sybil:~$ kate
^Z
[1]+  Arrêté                kate
maxence@Sybil:~$ jobs
[1]+  Arrêté                kate
maxence@Sybil:~$ fg %1
kate
^C
maxence@Sybil:~$ jobs
maxence@Sybil:~$ kate &
[1] 8367
maxence@Sybil:~$ okular &
[2] 8381
maxence@Sybil:~$ jobs
[1]-  En cours d'exécution   kate &
[2]+  En cours d'exécution   okular &
maxence@Sybil:~$ jobs
[1]-  En cours d'exécution   kate &
[2]+  En cours d'exécution   okular &
maxence@Sybil:~$ fg %2
okular
^C
maxence@Sybil:~$ jobs
[1]+  En cours d'exécution   kate &
			\end{lstlisting}
			shell.c :
			\begin{lstlisting}[frame=single,basicstyle=\footnotesize,language=bash]
shell> kate
^Zprocessus 8797 Suspendu
shell> jobs
[1] 	 Suspendu 	8797	 kate
shell> fg %1
processus 8797 mis au premier plan
[1] 	 En cours d'execution au premier plan	8797	 kate
^C
Fini 8797
delete 8797
shell> jobs
shell> kate &
shell> jobs 
[1] 	 En cours d'execution en arriere plan 	8809 	kate
shell> stop %1
processus 8809 Suspendu
shell> jobs
[1] 	 Suspendu 	8809	 kate
shell> exit
			\end{lstlisting}
\end{document}
